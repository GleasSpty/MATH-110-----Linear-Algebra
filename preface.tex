\chapter*{Preface}

\addtocontents{toc}{\protect\setcounter{tocdepth}{-1}}

I started writing these notes in order to prepare to teach MATH 110 Linear Algebra at the University of California, Berkeley during Summer 2017.  My obsessive-compulsive perfectionism and completionism turned them into what they are today.

At first, these notes were really just for me---I wanted to be sure I was ready to teach the class.  As I'm sure you're aware if you've ever taught before, there is much more to being able to teach well than simply knowing all the material.  For example, it is not enough to simply know Theorems 1 and 2.  Among other things, for example, you have to know the order in which they come in the theory.  Most of the motivation for starting these notes was to make sure I got all of that straight in my mind before I went up in front of a class.

\section*{A note to the reader}

With the exception of specific examples,\footnote{For example, I need to assume that we know what the real numbers are to even talk about $\R ^d$.} the mathematics in these notes is developed ``from the ground up''.  In particular, besides these exceptions, in principle, there are no prerequisites.  That said, there is a modest amount of basic material that I cover sufficiently fast that it would be very helpful if you had at least passing familiarity with.  Essentially all of this `prerequisite' material is given in the appendices.\footnote{Though there is also quite a bit of material there that I would not expect you to know.}  I recommend you read these notes linearly, and refer to the appendices as needed when you come across concepts you are not familiar with.  The notes are written in such a way that I would expect a student with no background to refer to the appendices \emph{very often} in the beginning, but very little by the end.

Of all the statements which are true in these notes, they are roughly divided into two broad categories:  the statements which are true by definition and the statements which are true because we can prove them.  For the former, we have \emph{definitions}; for the latter, we have \emph{theorems}, \emph{propositions}, \emph{corollaries}, \emph{lemmas}, and \emph{claims}.  We also have ``meta'' versions of (some of) these.

A definition is exactly what you think it is.  A ``meta-definition'' is actually a whole collection of definitions---whenever you plug something in for XYZ, you get an actual definition.  This explanation is probably not very lucid now, but I imagine it should be pretty obvious what I mean by this when you actually come to them.  For the record, ``meta-definition'' is not a standard term (and I don't really think there is a standard term for this).\footnote{The closest I can think of is \emph{axiom schema}, but ``definition schema'' sounds quite awkward to my ear.}  Similarly for ``'meta-propositions'', etc..

There is no hard and fast distinction between what I called theorems, propositions, corollaries, and lemmas.  I tried to roughly adhere to the following conventions.  If a result is used only in a proof of a single result and nowhere else, it is a \emph{lemma}.  If a result follows immediately or almost immediately from another result, it is a \emph{corollary}.  Results of particular significance are \emph{theorems}.  Everything else is a \emph{proposition}.  Claims, on the other hand, are distinct in that, not only only are they used in the proof of a single result like lemmas, but furthermore they wouldn't even make sense as stand-alone results (for example, if they use notation specific to the proof).

In particular, note that the distinction between theorems and propositions has to do with the relative \emph{significance}\footnote{Obviously this is completely subjective and I would not expect any mathematician to pick out the exact same results which deserve the title of ``theorem''.} of the \emph{statement} of the result, and has nothing to to with the \emph{difficulty} of the \emph{proof}.  Indeed, there are quite a few rather trivial results labeled as theorems simply because they are important.

There are also statements presented in blue boxes.  The blue box is meant to draw attention to the fact contained therein, as it is particularly important for one reason or another.  That said, the content in the blue box doesn't always contain the ``full story'', and is potentially a ``watered-down'' version of the truth.  For example, we will very often omit stating what things are in the box itself and the reader will have to consult the surrounding context to gain the full meaning.  This is by design---the boxes are supposed to highlight something important for convenience of the reader, not precise mathematics per se, and `bogging it down' with details that are probably clear from context defeats the purpose of being quick and convenient.

I should mention that every now and then I give nonstandard names to results which would otherwise not have names.  Part of the motivation for this is that I personally find this makes it easier to remember which result is which.  For example, would you rather I refer to ``\cref{FundamentalTheoremOfDiagonalizability}'' or the ``\nameref{FundamentalTheoremOfDiagonalizability}''?  Just be warned that you shouldn't go up to other mathematicians, use these names, and expect them to know what you're talking about.  (I will point it out when a name is nonstandard.)

As an unimportant comment, I mention in case you're curious that I used double quotes when I am quoting something, usually a term or phrase either I or people in general use, and single quotes to indicate that the thing is quotes is not literally that thing.  For example, mathematicians prove theorems and physicists prove `theorems'.

\blankline
\horizontalrule
\blankline

\hfill Jonathan Gleason \\
\hfill Ph.~D. Student, Department of Mathematics, University of California, Berkeley \\
\hfill First draft (of preface):  18 June 2017 \\
\hfill 16 June 2017

\addtocontents{toc}{\protect\setcounter{tocdepth}{3}}